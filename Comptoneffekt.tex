%! TeX program = lualatex
%---------------------------ALLGEMEINE IMPORTS-------------------------------------
\documentclass[12pt,english,ngerman]{scrartcl}
\input{./protokoll_template/template.latex/input/shared_preamble.tex}

% Kopfzeile
\ihead{WS22\\
	29.03.2023} \chead{\textsc{Stark} Matthias - 12004907 \\
	\textsc{Philipp} Maximilian - 11839611}
\ohead{Röntgenfluoreszenzanalyse \\ 
%FLAB 2 \\
Compton-Effekt}

% Fußzeile
%todo
%\addbibresource{SolarWirkungsgrad.bib}

\usepackage{luacode}

\DeclareSIUnit\px{px}
\DeclareSIUnit\strich{|||}
\DeclareSIUnit\Var{var}
\DeclareSIUnit\VA{VA}
\DeclareSIUnit\bar{bar}

\usepackage{cleveref}

\crefname{enumerate}{Aufzählung}{Aufzählungen}

\begin{document}

\begin{luacode*}
	dofile("createExtraPDF.lua")
\end{luacode*}

%todo
%\includepdf{./deckblatt.pdf}
\tableofcontents

\newpage

\section{Aufgabenstellung\label{Auf}}

\subsection{Röntgenfluoreszenzanalyse}

Im Zuge des Beispiels Röntgenfluoreszenzanalyse sind folgende Versuche durchzuführen:

\begin{itemize}
	\item Aufnahme und Kalibrierung eines Röntgenenergiespektrums
	\item Zeigen der Gültigkeit des Moseleyschen Gesetzes anhand der bereitgestellten Metalle und 
	Ermittlung der Abschirmkonstante der K-Linien
	\item Analyse der Zusammensetzung von unbekannten Proben
\end{itemize}

\subsection{Compton-Effekt}

Ziel dieses Versuchs ist die Messung der Energie der gestreuten Photonen in Abhängigkeit vom Streuwinkel. 
Dabei werden folgende Punkte durchgeführt:

\begin{itemize}
	\item Aufnahme des Primärspektrums und Energiekalibrierung des Detektors
	\item Aufnahme der Spektren in Streuanordnung
	\item Bestimmung der Energie als Funktion des Streuwinkels
\end{itemize}


\section{Grundlagen}\label{Grund}


\subsection{Röntgenfluoreszenzanalyse}


\subsection{Compton-Effekt}



\section{Versuchsanordnung}\label{sec:versuchsanordnung}

Für beide Teile des Versuchs wird das Röntgengerät aus \autoref{fig:aufabau} verwendet.
Zunächst wird die Stromversorgung
und die Verbindung zum PC mittels Cassy Lab hergestellt. Nun wird mit den Drehschaltern eine
Spannung von \SI{30.0(2)}{\kilo eV} und ein Strom von \SI{1.00(2)}{mA} eingestellt.
Nach richtigen SChließen der Tür wird die Röntgenstrahlung mit dem HV Knopf kurz eingeschaltet,
um zu überprüfen, ob das Gerät funktioniert.

\begin{figure}[H]
	\begin{center}
		\includegraphics[width =0.8\textwidth]{./figures/aufbau.jpg}
	\end{center}
	\caption[Verwendetes Röntgengerät]
	{Verwendetes Röntgengerät\\
	1 \dots Röherenraum mit Röntgenröhre\\
	2 \dots Kollimator\\
	3 \dots Targetarm mit Targettisch auf Goniometer\\
	4 \dots Sensor auf Goniometer\\
	5 \dots Bedienfeld mit Drehschalter für alle Einstellungen\\
	6 \dots Cassy Lab 2\\
	7 \dots Pinzette für Probenwechsel\\
	8 \dots Zr- Filter\\
	9 \dots Abschwächerblende
	}\label{fig:aufabau}
\end{figure}

\subsection{Röntgenfluoreszenzanalyse}

Zunächst werden die Abstände zwischen Kollimator und Target bzw. zwischen Target und Sensor eingestellt, damit diese 
5 - 6 cm betragen. Dies geschieht manuell durch Lockerung der entsprechenden Schraube und Bewegung des Bauteils auf den
Führungsschienen.
Nun werden die Winkel des Targetarms und 
des Senosors eingestellt. Wichtig ist dabei, dass dies elektrisch, über die entsprechenden Knöpfe geschieht.
Der Winkel des Targets wird dabei auf \SI{45.0(2)}{\degree} eingestellt, der des Sensors auf \SI{90.0(2)}{\degree}.
Mit der "Zero"-Taste können die Winkel zurück in die Ausgangsposition gebracht werden.

\subsection{Compton-Effekt}

Für diesen Teil des Versuchs wird der Zr-Filter (siehe 8 in \autoref{fig:aufabau}) verwendet.
Dieser wird über den Metalbolzen des Kolimators geschoben, bevor dieser in seine Einbuchtung gesteckt wird.
Für die Aufnahme des Kalibrierungsspektrums wird zusätzlich noch die Abschwächerblende verwendet. Diese wird 
über den Kolimator geschoben, wie in \autoref{fig:abschwacheblende} sichtbar. Zusätzlich wird für
die Kalibration auch der Targethalter ausgebaut.


\begin{figure}[H]
	\centering
	\captionbox[Montierte Abschwächerblende]{Montierte Abschwächerblende\label{fig:abschwacheblende}}{
		\includegraphics[width=.45\textwidth]{./figures/blende.PNG} } \hfill
	\captionbox[Maximale Auslenkung des Sensors]{Maximale Auslenkung des Sensors\label{fig:auslenkung}}{
		\includegraphics[width=0.45\textwidth]{./figures/auslenkung.PNG} }
\end{figure}


Für die Messung der Streuwinkel wir die Abschwächerblende wider entfernt und der Targethalter eingebaut. 
Bezüglich der Abstände von Sensor und Targethalter wird darauf geachtet, dass ein Schwenken von bis zu 
\SI{150.0(2)}{\degree} möglich ist, wie in \autoref{fig:auslenkung} sichtbar.


\section{Geräteliste}\label{sec:geraeteliste}

%todo

Für die Elektronen-Spin-Resonanz werden die in \autoref{tab:gerate_spin} aufgelisteten
Geräte verwendet.

\begin{table}[H]
	\caption{Verwendete Geräte für die Elektronen-Spin-Resonanz
	}
	\begin{tblr}{cells={font=\footnotesize},colspec={lllll}}
			\textbf{Gerätetyp}    & \textbf{Hersteller} & \textbf{Typ}     & \textbf{Inventar-Nr} & \textbf{Anmerkung}              \\
			ESR-Grundgerät        & KFU Graz            & Rep-Art-Onl-1066 & REP103801            &                                 \\
			ESR-Betriebsgerät     & Leybold             & 514571           &                      & mit Amperemeter                 \\
			Zweikanal Oszilloskop & Hameg               & HM205-2          & DOZ-3                & analog                          \\
			Helmholzspule         & LD                  & 555604           &                      & 2 x                             \\
			Steckspulen           &                     &                  &                      & mit unterschiedlichen Windungen \\
			Graphitprobe          &                     &                  &                      &                                 \\
			Sockel                &                     &                  &                      &                    \\
			Kabel                 &                     &                  &                      & BNC und Banane                 
	\end{tblr}\label{tab:gerate_spin}
\end{table}



\section{Versuchsdurchführung und Messergebnisse}\label{sec:versuchsdurchfuehrung_messergebnisse}

\subsection{Röntgenfluoreszenzanalyse}

\subsubsection{bereitgestellte Metalle}


\subsubsection{unbekannte Probe}


\subsection{Compton-Effekt}

\subsubsection{Energiekalibrierung des Detektors}

\subsubsection{Aufnahme der Spektren bei verschiedenen Winkeln}





\section{Auswertung}\label{sec:auswertung}

Um zu sehen wie sich die Unsicherheit der Messungen bis in die Ergebnisse
fortpflanzt, ist erweiterte Gauss-Methode verwendet worden. Die Grundlagen
dieser Methode stammen von den Powerpointfolien von
GUM~\cite{wolfgang_kessel_isobipm-gum_2004}. Für die Auswertung ist die
Progammiersprache Python im speziellen die Pakete \verb#labtool-ex2#,
\verb#pandas#, \verb#sympy#, \verb#lmfit# zur Hilfe genommen worden.
\verb#lmfit# wurde für das Fitten verwendet, \verb#sympy# wurde für symbolische
Manipulation verwendet und die restlichen Pakete für leichteres Handhaben der
Daten. Dies wurde aber alles durch \verb#labtool-ex2# abstrahiert.

Um höchstmögliche Genauigkeit zu garantieren wird erst bei der Darstellung der
Wert in Tabellen gerundet.


\subsection{Röntgenfluoreszenzanalyse}

\subsubsection{bereitgestellte Metalle}


\subsubsection{unbekannte Probe}


\subsection{Compton-Effekt}

\subsubsection{Energiekalibrierung des Detektors}

\subsubsection{Aufnahme der Spektren bei verschiedenen Winkeln}



\section{Diskussion}\label{sec:diskussion}

\subsection{Röntgenfluoreszenzanalyse}

\subsubsection{bereitgestellte Metalle}


\subsubsection{unbekannte Probe}


\subsection{Compton-Effekt}

\subsubsection{Energiekalibrierung des Detektors}

\subsubsection{Aufnahme der Spektren bei verschiedenen Winkeln}



\section{Zusammenfassung}\label{sec:zusammenfassung}

Hier werden nochmals alle Ergebnisse dieser Experimentenfolge aufgelistet.
Wobei die meisten zu erstellenden Diagramme Aufgrund der Länge der
\autoref{sec:auswertung} entnommen werden sollen.

\subsection{Röntgenfluoreszenzanalyse}

\subsubsection{bereitgestellte Metalle}


\subsubsection{unbekannte Probe}


\subsection{Compton-Effekt}

\subsubsection{Energiekalibrierung des Detektors}

\subsubsection{Aufnahme der Spektren bei verschiedenen Winkeln}


\newpage
\printbibliography
\listoffigures
\listoftables
\end{document}
