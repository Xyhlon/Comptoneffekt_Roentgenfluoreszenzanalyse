%! TeX program = lualatex
%---------------------------ALLGEMEINE IMPORTS-------------------------------------
\documentclass[12pt,english,ngerman]{scrartcl}
\input{./protokoll_template/template.latex/input/shared_preamble.tex}

% Kopfzeile
\ihead{WS22\\
	22.03.2023} \chead{\textsc{Stark} Matthias - 12004907 \\
	\textsc{Philipp} Maximilian - 11839611}
\ohead{FLAB 2 \\
	Festkörperphysik}

% Fußzeile
%todo
%\addbibresource{SolarWirkungsgrad.bib}

\usepackage{luacode}

\DeclareSIUnit\px{px}
\DeclareSIUnit\strich{|||}
\DeclareSIUnit\Var{var}
\DeclareSIUnit\VA{VA}
\DeclareSIUnit\bar{bar}

\usepackage{cleveref}

\crefname{enumerate}{Aufzählung}{Aufzählungen}

\begin{document}

\begin{luacode*}
	dofile("createExtraPDF.lua")
\end{luacode*}

%todo
%\includepdf{./deckblatt.pdf}
\tableofcontents

\newpage

\section{Aufgabenstellung\label{Auf}}

\subsection{Röntgenfluoreszenzanalyse}

Im Zuge des Beispiels Röntgenfluoreszenzanalyse sind folgende Versuche durchzuführen:

\begin{itemize}
	\item Aufnahme und Kalibrierung eines Röntgenenergiespektrums
	\item Zeigen der Gültigkeit des Moseleyschen Gesetzes anhand der bereitgestellten Metalle und 
	Ermittlung der Abschirmkonstante der K-Linien
	\item Analyse der Zusammensetzung von unbekannten Proben
\end{itemize}

\subsection{Compton-Effekt}

Ziel dieses Versuchs ist die Messung der Energie der gestreuten Photonen in Abhängigkeit vom Streuwinkel. 
Dabei werden folgende Punkte durchgeführt:

\begin{itemize}
	\item Aufnahme des Primärspektrums und Energiekalibrierung des Detektors
	\item Aufnahme der Spektren in Streuanordnung
	\item Bestimmung der Energie als Funktion des Streuwinkels
\end{itemize}


\section{Grundlagen}\label{Grund}

\subsection{Röntgenfluoreszenzanalyse}


\subsection{Compton-Effekt}

%\begin{figure}[H]
%	\begin{center}
%		\includegraphics[width =0.5\textwidth]{./figures/messung_s.PNG}
%	\end{center}
%	\caption[Von Auslenkung s auf Radius r schließen]
%	{Von Auslenkung s auf Radius r schließen \cite{unterlagen_welle_teilchen_dualismus}
%	}\label{fig:von_s_auf_r}
%\end{figure}

\section{Versuchsanordnung}\label{sec:versuchsanordnung}

\subsection{Röntgenfluoreszenzanalyse}


\subsection{Compton-Effekt}

\section{Geräteliste}\label{sec:geraeteliste}

Für die Elektronen-Spin-Resonanz werden die in \autoref{tab:gerate_spin} aufgelisteten
Geräte verwendet.

\begin{table}[H]
	\caption{Verwendete Geräte für die Elektronen-Spin-Resonanz
	}
	\begin{tblr}{cells={font=\footnotesize},colspec={lllll}}
			\textbf{Gerätetyp}    & \textbf{Hersteller} & \textbf{Typ}     & \textbf{Inventar-Nr} & \textbf{Anmerkung}              \\
			ESR-Grundgerät        & KFU Graz            & Rep-Art-Onl-1066 & REP103801            &                                 \\
			ESR-Betriebsgerät     & Leybold             & 514571           &                      & mit Amperemeter                 \\
			Zweikanal Oszilloskop & Hameg               & HM205-2          & DOZ-3                & analog                          \\
			Helmholzspule         & LD                  & 555604           &                      & 2 x                             \\
			Steckspulen           &                     &                  &                      & mit unterschiedlichen Windungen \\
			Graphitprobe          &                     &                  &                      &                                 \\
			Sockel                &                     &                  &                      &                    \\
			Kabel                 &                     &                  &                      & BNC und Banane                 
	\end{tblr}\label{tab:gerate_spin}
\end{table}

%todo max bitte schiebelehre ergänzen


\section{Versuchsdurchführung und Messergebnisse}\label{sec:versuchsdurchfuehrung_messergebnisse}

\subsection{Röntgenfluoreszenzanalyse}

\subsubsection{bereitgestellte Metalle}


\subsubsection{unbekannte Probe}


\subsection{Compton-Effekt}

\subsubsection{Energiekalibrierung des Detektors}

\subsubsection{Aufnahme der Spektren bei verschiedenen Winkeln}





\section{Auswertung}\label{sec:auswertung}

Um zu sehen wie sich die Unsicherheit der Messungen bis in die Ergebnisse
fortpflanzt, ist erweiterte Gauss-Methode verwendet worden. Die Grundlagen
dieser Methode stammen von den Powerpointfolien von
GUM~\cite{wolfgang_kessel_isobipm-gum_2004}. Für die Auswertung ist die
Progammiersprache Python im speziellen die Pakete \verb#labtool-ex2#,
\verb#pandas#, \verb#sympy#, \verb#lmfit# zur Hilfe genommen worden.
\verb#lmfit# wurde für das Fitten verwendet, \verb#sympy# wurde für symbolische
Manipulation verwendet und die restlichen Pakete für leichteres Handhaben der
Daten. Dies wurde aber alles durch \verb#labtool-ex2# abstrahiert.

Um höchstmögliche Genauigkeit zu garantieren wird erst bei der Darstellung der
Wert in Tabellen gerundet.


\subsection{Röntgenfluoreszenzanalyse}

\subsubsection{bereitgestellte Metalle}


\subsubsection{unbekannte Probe}


\subsection{Compton-Effekt}

\subsubsection{Energiekalibrierung des Detektors}

\subsubsection{Aufnahme der Spektren bei verschiedenen Winkeln}



\section{Diskussion}\label{sec:diskussion}

\subsection{Röntgenfluoreszenzanalyse}

\subsubsection{bereitgestellte Metalle}


\subsubsection{unbekannte Probe}


\subsection{Compton-Effekt}

\subsubsection{Energiekalibrierung des Detektors}

\subsubsection{Aufnahme der Spektren bei verschiedenen Winkeln}



\section{Zusammenfassung}\label{sec:zusammenfassung}

Hier werden nochmals alle Ergebnisse dieser Experimentenfolge aufgelistet.
Wobei die meisten zu erstellenden Diagramme Aufgrund der Länge der
\autoref{sec:auswertung} entnommen werden sollen.

\subsection{Röntgenfluoreszenzanalyse}

\subsubsection{bereitgestellte Metalle}


\subsubsection{unbekannte Probe}


\subsection{Compton-Effekt}

\subsubsection{Energiekalibrierung des Detektors}

\subsubsection{Aufnahme der Spektren bei verschiedenen Winkeln}


\newpage
\printbibliography
\listoffigures
\listoftables
\end{document}
