%! TeX program = lualatex
%---------------------------ALLGEMEINE IMPORTS-------------------------------------
\documentclass[12pt,english,ngerman]{scrartcl}
\input{./protokoll_template/template.latex/input/shared_preamble.tex}

% Kopfzeile
\ihead{WS22\\
	29.03.2023} \chead{\textsc{Stark} Matthias - 12004907 \\
	\textsc{Philipp} Maximilian - 11839611}
\ohead{Röntgenfluoreszenzanalyse \\
	%FLAB 2 \\
	Compton-Effekt}

% Fußzeile
%todo
%\addbibresource{SolarWirkungsgrad.bib}

\usepackage{luacode}

\DeclareSIUnit\px{px}
\DeclareSIUnit\strich{|||}
\DeclareSIUnit\Var{var}
\DeclareSIUnit\VA{VA}
\DeclareSIUnit\bar{bar}

\usepackage{cleveref}

\crefname{enumerate}{Aufzählung}{Aufzählungen}

\begin{document}

\begin{luacode*}
	dofile("createExtraPDF.lua")
\end{luacode*}

%todo
%\includepdf{./deckblatt.pdf}
\tableofcontents

\newpage

\section{Aufgabenstellung\label{Auf}}

\subsection{Röntgenfluoreszenzanalyse}

Im Zuge des Beispiels Röntgenfluoreszenzanalyse sind folgende Versuche
durchzuführen:

\begin{itemize}
	\item Aufnahme und Kalibrierung eines Röntgenenergiespektrums
	\item Zeigen der Gültigkeit des Moseleyschen Gesetzes anhand der bereitgestellten
	      Metalle und Ermittlung der Abschirmkonstante der K-Linien
	\item Analyse der Zusammensetzung von unbekannten Proben
\end{itemize}

\subsection{Compton-Effekt}

Ziel dieses Versuchs ist die Messung der Energie der gestreuten Photonen in
Abhängigkeit vom Streuwinkel. Dabei werden folgende Punkte durchgeführt:

\begin{itemize}
	\item Aufnahme des Primärspektrums und Energiekalibrierung des Detektors
	\item Aufnahme der Spektren in Streuanordnung
	\item Bestimmung der Energie als Funktion des Streuwinkels
\end{itemize}

\section{Grundlagen}\label{Grund}

\subsection{Röntgenfluoreszenzanalyse}

%todo Max text von röntgenflureszens bitte mit KI

\begin{figure}[H]
	\begin{center}
		\includegraphics[width =0.5\textwidth]{./figures/skizze_roentgendetektor.PNG}
	\end{center}
	\caption[Schematische Skizze eines Röntgendetektors] {Schematische Skizze eines
		Röntgendetektors \cite{unterlagen_rontgenfluorenzenz}
	}\label{fig:skizze_roetgendetektor}
\end{figure}

\begin{figure}[H]
	\begin{center}
		\includegraphics[width =0.5\textwidth]{./figures/termschema_roentgenlinien.PNG}
	\end{center}
	\caption[Termschema der charakteristischen Röentgenlinien] {Termschema der
		charakteristischen Röentgenlinien \cite{unterlagen_rontgenfluorenzenz}
	}\label{fig:termschema_roentgenlinien}
\end{figure}

\subsection{Compton-Effekt}

Der Comptonefferkt dient als Nachweis des Teilchencharakters von Licht. Er
besagt, wenn ein beliebiges Material mit Röntgenstrahlung der Wellenlänge
$\lambda_0$ bestrahlt, so kann bei der Streustrahlung eine Wellenlänge
$\lambda_s > \lambda_0$ festgestellt werden. Dies kann durch Stoßvorgänge
zwischen den Röntgenquanten und den freien Elektronen erklärt werden, bei denen
Energie vom Röntgen-Quant auf das Elektron übertragen wird, wie in
\autoref{fig:skizze_compton} sichtbar.

\begin{figure}[H]
	\begin{center}
		\includegraphics[width =\textwidth]{./figures/skizze_compton.PNG}
	\end{center}
	\caption[Darstellung des Comptoneffekts] {Schematische Darstellung des Comptoneffekts
		(b) Darstellung des Impulsübertrags \cite{unterlagen_compton}
	}\label{fig:skizze_compton}
\end{figure}

Zunächst wird das Elektron $e^{-}$ vor dem Stoß als ruhend betrachten. Es wird
davon ausgegangen, dass jenes im Streumaterial nur schwach gebunden ist.
Dadurch kann die Vernachlässigung der Bindungsenergie gerechtfertigt werden.
Beim einem Stoß kann nun aus der Erhaltung für die Energie $E$ und den Impuls
$p$ der Röntgenstrahlung, folgender Zusammenhang gefolgert werden.

\begin{equation}
	h v_0+e^{-} \rightarrow h v_S+e^{-}\left(E_{k i n}\right)
\end{equation}
\begin{equation}
	E=h v=\frac{h c}{\lambda},\  \vec{p}=\hbar \vec{k} \text { und }|p|=\frac{h}{\lambda}
	\label{eq:energie_impuls}
\end{equation}

$h$ beschreibt dabei das Plancksches Wirkungsquantum,
$c$ die Lichtgeschwindigkeit,
$\lambda$ die Wellenlänge und
$v$ die Frequenz der Röntgenstrahlung

Eine relativistische Betrachtung liefert für den Energiesatz somit:
\begin{equation}
	E_{\text {kin }}^e=\frac{m_0 c^2}{\sqrt{1-\beta^2}}-m_0 c^2, \text { mit } \beta=\frac{v}{c}
	\label{eq:energiesatz_relativistisch}
\end{equation}

Daraus folgt für den Impuls:
\begin{equation}
	\hbar \overrightarrow{k_0}=\hbar \overrightarrow{k_S}+\overrightarrow{p_e} \text { mit } \overrightarrow{p_e}=\frac{m_0 \overrightarrow{v}}{\sqrt{1-\beta^2}}
	\label{eq:impuls_relativistisch}
\end{equation}

Daraus folgt:

\begin{equation}
	v_0-v_S=\frac{h}{m_0 c^2} v_0 v_S(1-\cos \varphi)
	\label{eq:v0-vs}
\end{equation}

Woraus sich für die Compton-Streuformel folgender Zusammenhang ergibt:

\begin{equation}
	\lambda_S-\lambda_0=2 \lambda_C \sin ^2 \varphi_2 \text { mit } \lambda_C=\frac{\hbar}{m_0 \mathrm{C}}=2,4262 \cdot 10^{-12} \mathrm{~m}
	\label{eq:compton_streuformel}
\end{equation}

$\lambda_C$ bezeichnet dabei die Compton-Wellenlänge des Elektrons.

Im Versuch werden die Energien der gestreuten Strahlung gemessen. Die
entsprechend umgeformte Formel lautet:
\begin{equation}
	E_S=\frac{E_0}{1+\frac{E_0}{m_0 c^2}(1-\cos \varphi)}
	\label{eq:energie_der_strahlung}
\end{equation}

$E_s$ bezeichnet dabei die Energie der gestreuten Strahlung und
$E_0$ die Primärenergie \cite{unterlagen_compton}.

\section{Versuchsanordnung}\label{sec:versuchsanordnung}

Für beide Teile des Versuchs wird das Röntgengerät aus \autoref{fig:aufabau}
verwendet. Zunächst wird die Stromversorgung und die Verbindung zum PC mittels
Cassy Lab hergestellt. Nun wird mit den Drehschaltern eine Spannung von
\SI{30.0(2)}{\kilo eV} und ein Strom von \SI{1.00(2)}{mA} eingestellt. Nach
richtigen SChließen der Tür wird die Röntgenstrahlung mit dem HV Knopf kurz
eingeschaltet, um zu überprüfen, ob das Gerät funktioniert.

\begin{figure}[H]
	\begin{center}
		\includegraphics[width =0.8\textwidth]{./figures/aufbau.jpg}
	\end{center}
	\caption[Verwendetes Röntgengerät] {Verwendetes Röntgengerät                                   \\
		1 \dots Röherenraum mit Röntgenröhre                       \\
		2 \dots Kollimator                                         \\
		3 \dots Targetarm mit Targettisch auf Goniometer           \\
		4 \dots Sensor auf Goniometer                              \\
		5 \dots Bedienfeld mit Drehschalter für alle Einstellungen \\
		6 \dots Cassy Lab 2                                        \\
		7 \dots Pinzette für Probenwechsel                         \\
		8 \dots Zr- Filter                                         \\
		9 \dots Abschwächerblende
	}\label{fig:aufabau}
\end{figure}

\subsection{Röntgenfluoreszenzanalyse}

Zunächst werden die Abstände zwischen Kollimator und Target bzw. zwischen
Target und Sensor eingestellt, damit diese 5 - 6 cm betragen. Dies geschieht
manuell durch Lockerung der entsprechenden Schraube und Bewegung des Bauteils
auf den Führungsschienen. Nun werden die Winkel des Targetarms und des Senosors
eingestellt. Wichtig ist dabei, dass dies elektrisch, über die entsprechenden
Knöpfe geschieht. Der Winkel des Targets wird dabei auf \SI{45.0(2)}{\degree}
eingestellt, der des Sensors auf \SI{90.0(2)}{\degree}. Mit der "Zero"-Taste
können die Winkel zurück in die Ausgangsposition gebracht werden. Die
verwendeten Proben sind in \autoref{fig:proben} sichtbar.

\begin{figure}[H]
	\begin{center}
		\includegraphics[width =\textwidth]{./figures/proben.jpg}
	\end{center}
	\caption[Verwendete Proben] {Verwendete Proben            \\
		1 \dots Titan-Probe (Ti)     \\
		2 \dots Eisen-Probe (Fe)     \\
		3 \dots Nickel-Probe (Ni)    \\
		4 \dots Kupfer-Probe (Cu)    \\
		5 \dots Zink-Probe (Zn)      \\
		6 \dots Zirkonium-Probe (Zr) \\
		7 \dots Molybdän-Probe (Mo)  \\
		8 \dots Silber-Probe (Ag)    \\
		9 \dots Plexiglasplatte      \\
		10 \dots Pinzette für Probenwechsel
	}\label{fig:proben}
\end{figure}

\subsection{Compton-Effekt}

Für diesen Teil des Versuchs wird der Zr-Filter (siehe 8 in
\autoref{fig:aufabau}) verwendet. Dieser wird über den Metalbolzen des
Kolimators geschoben, bevor dieser in seine Einbuchtung gesteckt wird. Für die
Aufnahme des Kalibrierungsspektrums wird zusätzlich noch die Abschwächerblende
verwendet. Diese wird über den Kolimator geschoben, wie in
\autoref{fig:abschwacheblende} sichtbar. Zusätzlich wird für die Kalibration
auch der Targethalter ausgebaut.

\begin{figure}[H]
	\centering
	\captionbox[Montierte Abschwächerblende]{Montierte Abschwächerblende\label{fig:abschwacheblende}}{
		\includegraphics[width=.45\textwidth]{./figures/blende.PNG} } \hfill
	\captionbox[Maximale Auslenkung des Sensors]{Maximale Auslenkung des Sensors\label{fig:auslenkung}}{
		\includegraphics[width=0.45\textwidth]{./figures/auslenkung.PNG} }
\end{figure}

Für die Messung der Streuwinkel wir die Abschwächerblende wider entfernt und
der Targethalter eingebaut. Bezüglich der Abstände von Sensor und Targethalter
wird darauf geachtet, dass ein Schwenken von bis zu \SI{150.0(2)}{\degree}
möglich ist, wie in \autoref{fig:auslenkung} sichtbar.

\section{Geräteliste}\label{sec:geraeteliste}

Für den Versuch werden die in \autoref{tab:gerate} aufgelisteten Geräte und für
die Röntgenfluoreszenzanalyse die Proben aus \autoref{tab:proben} verwendet.

\begin{table}[H]
	\begin{center}
		\caption{Verwendete Geräte für den Versuch
		}
		\begin{tblr}{cells={font=\footnotesize},colspec={lllll}}
			\textbf{Gerätetyp} & \textbf{Hersteller} & \textbf{Typ} & \textbf{Inventar-Nr} \\
			Röntgengerät       & LD                  & 554800       & 310082130000         \\
			Röntgendetektor    & LD                  & 559938       & 300025970000         \\
			VKA-Box            & LD                  &              & 300025980000         \\
			Röntgenröhre       & LD                  & Molybdän     & 310094400000         \\
			Kolimator          &                     &              &                      \\
			Abschwächerblende  &                     &              &                      \\
			Cassy Lab 2        & LD                  & 524013       &                      \\
			Computersoftware   & Cassy Lab           &              &                      \\
			Zr Filter          &                     &              &                      \\
			Plexiglasplatte    &                     &              &                      \\
			Pinzette           &                     &              &
		\end{tblr}\label{tab:gerate}
	\end{center}
\end{table}

\begin{table}[H]
	\begin{center}
		\caption{Proben für den Versuch der Röntgenfluoreszenzanalyse
		}
		\begin{tblr}{cells={font=\footnotesize},colspec={lllll}}
			\textbf{Probe} & \textbf{Element} & \textbf{Anmerkung} \\
			Fe/Zn          & Verzinktes Eisen & Kalibrierung       \\
			Ti             & Titan            &                    \\
			Fe             & Eisen            &                    \\
			Ni             & Nickel           &                    \\
			Cu             & Kupfer           &                    \\
			Zn             & Zink             &                    \\
			Zr             & Zirkonium        &                    \\
			Mo             & Molybdän         &                    \\
			Ag             & Silber           &                    \\
			Neodyn-Magnet  &                  &                    \\
			Ring           &                  &                    \\
		\end{tblr}\label{tab:proben}
	\end{center}
\end{table}

\section{Versuchsdurchführung und Messergebnisse}\label{sec:versuchsdurchfuehrung_messergebnisse}

\subsection{Röntgenfluoreszenzanalyse}

\subsubsection{Aufnahme des Kalibrierungsspektrums}

Nachdem der Versuchsaufbau, wie bereits in \autoref{sec:versuchsanordnung}
erklärt, durchgeführt wurde, wird zunächst das Kalibrierungstarget aus die
entsprechende Position gelegt. Dabei handelt es sich um ein verzinktes
Eisenblech. Nun werden die Winkel des Targets und des Sensors richtig
eingestellt und der Sensor über das Cassy Lab 2 mit dem Computer verbunden. In
der Computersoftware werden unter den Messparametern die Einstellungen
"Vielkanalmessung, 512 Kanäle, negative Pulse, Verstärkung = -2,5, Messdauer =
300 s“ eingegeben. Nun kann der Röntgenstrom eingeschaltet und die Messung
gestartet werden.

Nun müssen den zwei erzeugten Peaks die entsprechenden Energien zugeordnet
werden. Für Fe ist die entsprechende Energie \SI{6.40}{keV} und für Zn
\SI{8.64}{keV} \cite{unterlagen_rontgenfluorenzenz}. Das erzeugte
Kalibrierungsspektrum, samt den durchgeführten Einstellungen, ist in
\autoref{fig:kalibrierung_rontgenfluoreszenz} sichtbar.

\begin{figure}[H]
	\begin{center}
		\includegraphics[width =\textwidth]{./figures/roentgen/FeZnKalibrierung.PNG}
	\end{center}
	\caption[Aufgezeichnetes Kalibrierungsspektrum mit Interface] {\footnotesize
		Aufgezeichnetes Kalibrierungsspektrum von Fe und Zn mit Interface \\
		$E_A$ \dots entsprechende Energie in keV                          \\
		$N_A$ \dots verzeichnete Counts
	}\label{fig:kalibrierung_rontgenfluoreszenz}
\end{figure}

\subsubsection{bereitgestellte Metalle}

Nachdem das Kalibrierungsspektrum aufgenommen wurde werden nun die
verschiedenen Proben auf den Targethalter gelegt und die Messung, wie bereits
zuvor für die Kalibreirung beschrieben, durchgeführt. Beim Wechseln der Proben
ist zu beachten, dass die Röntgenstrahlung immer ausgeschalten ist, wenn die
Tür geöffnet wird. Auch muss die Röntgenstrahlung kurz eingeschaltet werden bis
die Glühkathode einen konstanten Farbverlauf aufweißt, bevor die Messung
gestartet wird. Die so erzeugten Diagramme der einzenen Proben sind in
folgenden Abbildungen immer mit dem Referenzspektrum sichtbar.

\begin{figure}[H]
	\centering
	\captionbox[Gemessenes Spektrum der Ti - Probe]{\footnotesize Gemessenes Spektrum der Ti - Probe (rot) mit
		Kalibrierungsspektrum von Fe und Zn (schwarz) \\
		$E_A$ \dots entsprechende Energie in keV\\
		$N_A$ \dots verzeichnete Counts
		\label{fig:spektrum_Ti}}{
		\includegraphics[width=.45\textwidth]{./figures/roentgen/TI.png} } \hfill
	\captionbox[Gemessenes Spektrum der Fe - Probe]{\footnotesize Gemessenes Spektrum der Fe - Probe (rot) mit
		Kalibrierungsspektrum von Fe und Zn (schwarz) \\
		$E_A$ \dots entsprechende Energie in keV\\
		$N_A$ \dots verzeichnete Counts
		\label{spektrum_Fe}}{
		\includegraphics[width=0.45\textwidth]{./figures/roentgen/Fe.png} }
\end{figure}

\begin{figure}[H]
	\centering
	\captionbox[Gemessenes Spektrum der Ni - Probe]{\footnotesize Gemessenes Spektrum der Ni - Probe (rot) mit
		Kalibrierungsspektrum von Fe und Zn (schwarz) \\
		$E_A$ \dots entsprechende Energie in keV\\
		$N_A$ \dots verzeichnete Counts
		\label{fig:spektrum_Ni}}{
		\includegraphics[width=.45\textwidth]{./figures/roentgen/Ni.png} } \hfill
	\captionbox[Gemessenes Spektrum der Cu - Probe]{\footnotesize Gemessenes Spektrum der Cu - Probe (rot) mit
		Kalibrierungsspektrum von Fe und Zn (schwarz) \\
		$E_A$ \dots entsprechende Energie in keV\\
		$N_A$ \dots verzeichnete Counts
		\label{spektrum_Cu}}{
		\includegraphics[width=0.45\textwidth]{./figures/roentgen/Cu.png} }
\end{figure}

\begin{figure}[H]
	\centering
	\captionbox[Gemessenes Spektrum der Zn - Probe]{\footnotesize Gemessenes Spektrum der Zn - Probe (rot) mit
		Kalibrierungsspektrum von Fe und Zn (schwarz) \\
		$E_A$ \dots entsprechende Energie in keV\\
		$N_A$ \dots verzeichnete Counts
		\label{fig:spektrum_Zn}}{
		\includegraphics[width=.45\textwidth]{./figures/roentgen/Zn.png} } \hfill
	\captionbox[Gemessenes Spektrum der Zr - Probe]{\footnotesize Gemessenes Spektrum der Zr - Probe (rot) mit
		Kalibrierungsspektrum von Fe und Zn (schwarz) \\
		$E_A$ \dots entsprechende Energie in keV\\
		$N_A$ \dots verzeichnete Counts
		\label{spektrum_Zr}}{
		\includegraphics[width=0.45\textwidth]{./figures/roentgen/Zr.png} }
\end{figure}

\begin{figure}[H]
	\centering
	\captionbox[Gemessenes Spektrum der Mo - Probe]{\footnotesize Gemessenes Spektrum der Mo - Probe (rot) mit
		Kalibrierungsspektrum von Fe und Zn (schwarz) \\
		$E_A$ \dots entsprechende Energie in keV\\
		$N_A$ \dots verzeichnete Counts
		\label{fig:spektrum_Mo}}{
		\includegraphics[width=.45\textwidth]{./figures/roentgen/Mo.png} } \hfill
	\captionbox[Gemessenes Spektrum der Ag - Probe]{\footnotesize Gemessenes Spektrum der Ag - Probe (rot) mit
		Kalibrierungsspektrum von Fe und Zn (schwarz) \\
		$E_A$ \dots entsprechende Energie in keV\\
		$N_A$ \dots verzeichnete Counts
		\label{spektrum_Ag}}{
		\includegraphics[width=0.45\textwidth]{./figures/roentgen/Ag.png} }
\end{figure}

Der gemessene Energiewert des Peaks wird in der Software mit der paek finder
Funktion bestimmt und nochmals in \autoref{tab:energien_roentgen} angeführt.

\begin{table}
	\caption{}
	\label{tab:energien_roentgen}
	\centering
	\begin{tblr}{colspec={S[table-format=1.2(1)e2]S[table-format=2.1(1)]}}
{{{$E_\text{Char}$ / \si{\kilo\eV}}}} & {{{$Z$ / 1}}}\\
7.5(4)e-16 & 22.0(0)\\
1.03(4)e-15 & 26.0(0)\\
1.19(4)e-15 & 28.0(0)\\
1.28(4)e-15 & 29.0(0)\\
1.38(4)e-15 & 30.0(0)\\
2.50(4)e-15 & 40.0(0)\\
2.77(4)e-15 & 42.0(0)\\
3.48(4)e-15 & 47.0(0)\\
\end{tblr}

\end{table}

\begin{table}
	\caption{}
	\label{tab:energien_roentgen2}
	\centering
	\begin{tblr}{colspec={S[table-format=1.2(1)e2]S[table-format=2.1(1)]}}
{{{$E_\text{S}$ / \si{\kilo\eV}}}} & {{{$Z$ / 1}}}\\
1.31(4)e-15 & 28.0(0)\\
1.39(4)e-15 & 29.0(0)\\
1.52(4)e-15 & 30.0(0)\\
2.80(4)e-15 & 40.0(0)\\
3.11(4)e-15 & 42.0(0)\\
3.91(4)e-15 & 47.0(0)\\
\end{tblr}

\end{table}

\subsubsection{unbekannte Probe}

Nun werden zwei eigene Proben, sichtbar in \autoref{fig:unbekannte_pr}, auf den
Targethalter gelegt, um deren genaue Zusammensetzung zu bestimmen.

\begin{figure}[H]
	\begin{center}
		\includegraphics[width =0.5\textwidth]{./figures/eigene_proben.jpg}
	\end{center}
	\caption[Eigene Proben] {Eigene Proben                           \\
		1 \dots Neodynmagnet auf Eisenplättchen \\
		2 \dots Ring
	}\label{fig:unbekannte_pr}
\end{figure}

Um den Neodynmagneten besser auf dem Targethalter positionieren zu können,
wurde dieser auf die Eisenplatte gegeben. Bei der Auswertung ist nun zu
beachten, dass diese Eisenplatte auch berücksichtigt werden muss. Beim Ring ist
zu Beachten dass bei der Positionierung darauf geachtet wurde, dass die
Röntgenstrahlen den Ring auch wirklich treffen.

Die Messung der beiden Proben erfolgt nach dem gleichen Schema wie zuvor bei
den Metallplättchen. Die erzeugten Spektren sind in folgenden
\autoref{fig:spektrum_magnet} und \autoref{fig:spektrum_ring} sichtbar.

\begin{figure}[H]
	\centering
	\captionbox[Gemessenes Spektrum des Magneten]{\footnotesize Gemessenes Spektrum des Magneten (rot) mit
		Kalibrierungsspektrum von Fe und Zn (schwarz) \\
		$E_A$ \dots entsprechende Energie in keV\\
		$N_A$ \dots verzeichnete Counts
		\label{fig:spektrum_magnet}}{
		\includegraphics[width=.45\textwidth]{./figures/roentgen/Magnet.png} } \hfill
	\captionbox[Gemessenes Spektrum des Rings]{\footnotesize Gemessenes Spektrum des Rings (rot) mit
		Kalibrierungsspektrum von Fe und Zn (schwarz) \\
		$E_A$ \dots entsprechende Energie in keV\\
		$N_A$ \dots verzeichnete Counts
		\label{fig:spektrum_ring}}{
		\includegraphics[width=0.45\textwidth]{./figures/roentgen/Ring.png} }
\end{figure}

\subsection{Compton-Effekt}

\subsubsection{Energiekalibrierung des Detektors}

Nachdem das Röntgengerät wie bereits in \autoref{sec:versuchsanordnung}
beschrieben vorbereitet wurde, indem der Targetarm ausgebaut wird und Zr -
Filter, sowie die Abschwächerblende positioniert wurden, wird zunächst der
Emissionsstrom auf \SI{0.05(2)}{\milli\ampere} reduziert. In der Cassy Lab
Software werden die folgende Einstellungen vorgenommen: "Vielkanalmessung, 512
Kanäle, negative Pulse, Verstärkung = -4, Messdauer = 300 s“ Nun wird das
Maximum der Zählrate bestimmt, indem eine Messung gestartet wird und der Sensor
langsam um die 0°-Linie geschwenkt wird, bis das Maximum verzeichnet wird, was
im konkreten Fall bei folgenden Winkel $\alpha_{max}$ der Fall war.

\begin{equation*}
	\alpha_{max} = \SI{-0.6(2)}{\degree}
\end{equation*}

Mit diesem Winkel wird eine gesammte Messperiode aufgezeichnet. Auch hier ist
es wichtig darauf zu achten, dass die Tür ordnungsgemäß geschlossen ist. Nun
wird in der Computersoftware unter dem gleichnamigen Dialogfenster eine
Energiekalibrierung durchgeführt, indem die Energien für Au L$\alpha$ -
\SI{9.71}{keV} und Mo K$\alpha$ - \SI{17.44}{keV} übergeben werden
\cite{unterlagen_compton}. Das so erzeugte Kalibrierungsspektrum mit den
entsprechenden Einstellungen ist in \autoref{fig:kalibrierung_compton}
sichtbar.

\begin{figure}[H]
	\begin{center}
		\includegraphics[width =\textwidth]{./figures/compton/kalibrierung.png}
	\end{center}
	\caption[Aufgezeichnetes Kalibrierungsspektrum von Zr mit Interface] {\footnotesize
		Aufgezeichnetes Kalibrierungsspektrum von Zr Filter mit Interface \\
		$E_A$ \dots entsprechende Energie in keV                          \\
		$N_A$ \dots verzeichnete Counts
	}\label{fig:kalibrierung_compton}
\end{figure}

\subsubsection{Aufnahme der Spektren bei verschiedenen Winkeln}

Nun wird, wie bereits in \autoref{sec:versuchsanordnung} beschrieben, die
Abschwächerblende entfernt und der Targethalter wieder Eingebaut. Auf diesen
wirdvorsichtig die Plexiglasprobe gelegt. Weiters wird wieder ein
Emissionsstrom von \SI{1.00(2)}{\milli\ampere} verwendet. Nun wird der
Targetarm über den entsprechenden Drehknopf um einen Winkel von
\SI{20.0(2)}{\degree} geneigt und für alle Messungen in dieser Position
belassen. Für die verschiedenen Messungen wird nun der Neigungswinkel des
Sensors variiert. Um den tatsächlichen Auslenkungswinkel im Bezug zur
Energiekalibrierung zu messen muss beachtet werden, dass der zuvor bestimmte
Versatz von \SI{-0.6(2)}{\degree} immer berücksichtigt wird. Eine tatsächliche
Verdrehung des Sensors von \SI{30}{\degree} würde daher beispielsweise einem
Wert von \SI{29.4}{\degree} in der Anzeige entsprechen. Der besseren Übersicht
halber, werden die Winkel in \autoref{tab:werte_compton} bezüglich der
Kalibrierung angegeben. Nun wird in der Computersoftware über die peak-finder
Funktion die Energie des Peaks beim entsprechenden Abstrahlwinkel bestimmt und
in \autoref{tab:werte_compton} notiert.

\begin{table}
	\caption{}
	\label{tab:werte_compton}
	\centering
	\begin{tblr}{colspec={S[table-format=3.2(2)]S[table-format=2.2(2)]}}
{{{$\phi$ / \si{\degree}}}} & {{{$E_\text{Char}$ / \si{\kilo\eV}}}}\\
30.00(10) & 17.40(17)\\
45.00(10) & 17.3(3)\\
60.00(10) & 17.2(2)\\
75.00(10) & 17.1(3)\\
90.00(10) & 16.92(18)\\
105.00(10) & 16.8(3)\\
120.00(10) & 16.65(19)\\
135.00(10) & 16.5(3)\\
150.00(10) & 16.4(3)\\
\end{tblr}

\end{table}

Um die Verschiebung des Peaks klar sichtbar zu machen, wird in
\autoref{fig:verschiebung_compton} das gemessene Spektrum für einen Winkel des
Sensor von \SI{30.0(2)}{\degree} jenem von \SI{150.0(2)}{\degree}
gegenübergestellt.

\begin{figure}[H]
	\begin{center}
		\includegraphics[width =\textwidth]{./figures/compton/vergleich.png}
	\end{center}
	\caption[Aufgezeichnetes Kalibrierungsspektrum mit Interface] {\footnotesize Vergleich
		der Spektren bei unterschiedlichen Winkeln des Sensors \\
		Blau \dots Auslenkung des Sensors bezüglich der Kalibrierung um
		\SI{30.0(2)}{\degree}                                  \\
		Rot \dots Auslenkung des Sensors bezüglich der Kalibrierung um
		\SI{150.0(2)}{\degree}                                 \\
		$E_A$ \dots entsprechende Energie in keV               \\
		$N_A$ \dots verzeichnete Counts
	}\label{fig:verschiebung_compton}
\end{figure}

\section{Auswertung}\label{sec:auswertung}

Um zu sehen wie sich die Unsicherheit der Messungen bis in die Ergebnisse
fortpflanzt, ist die erweiterte Gauss-Methode verwendet worden. Die Grundlagen
dieser Methode stammen von den Powerpointfolien von
GUM~\cite{wolfgang_kessel_isobipm-gum_2004}. Für die Auswertung ist die
Progammiersprache Python im speziellen die Pakete \verb#labtool-ex2#,
\verb#pandas#, \verb#sympy#, \verb#lmfit# zur Hilfe genommen worden.
\verb#lmfit# wurde für das Fitten verwendet, \verb#sympy# wurde für symbolische
Manipulation verwendet und die restlichen Pakete für leichteres Handhaben der
Daten. Dies wurde aber alles durch \verb#labtool-ex2# abstrahiert.

Um höchstmögliche Genauigkeit zu garantieren wird erst bei der Darstellung der
Wert in Tabellen gerundet.

\subsection{Röntgenfluoreszenzanalyse}

\subsubsection{bereitgestellte Metalle}

\subsubsection{unbekannte Probe}

\subsection{Compton-Effekt}

\subsubsection{Energiekalibrierung des Detektors}

\subsubsection{Aufnahme der Spektren bei verschiedenen Winkeln}

\section{Diskussion}\label{sec:diskussion}

\subsection{Röntgenfluoreszenzanalyse}

\subsubsection{bereitgestellte Metalle}

\subsubsection{unbekannte Probe}

Um sicherzustellen, ob sich die Kalibration nicht verändert hat wurde zum
Schluss nochmals die Fe Probe eingelegt und das in \autoref{fig:fe_zumSchluss}
sichtbare Spektrum erzeugt.

\begin{figure}[H]
	\begin{center}
		\includegraphics[width =\textwidth]{./figures/roentgen/KalibrieungErhaltenFe.png}
	\end{center}
	\caption[Erneute Messung des Spektrums der Fe - Probe] {Erneute Messung des Spektrums
		der Fe - Probe (blau) mit zuvor erzeugtem Kalibrierungsspektrum von Fe und Zn
		(schwarz)                                \\
		$E_A$ \dots entsprechende Energie in keV \\
		$N_A$ \dots verzeichnete Counts
	}\label{fig:fe_zumSchluss}
\end{figure}

Hier wird klar ersichtlich, dass sich die Kalibrierung, wie erwartet, nicht
geändert hat.

\subsection{Compton-Effekt}

\subsubsection{Energiekalibrierung des Detektors}

\subsubsection{Aufnahme der Spektren bei verschiedenen Winkeln}

\section{Zusammenfassung}\label{sec:zusammenfassung}

Hier werden nochmals alle Ergebnisse dieser Experimentenfolge aufgelistet.
Wobei die meisten zu erstellenden Diagramme Aufgrund der Länge der
\autoref{sec:auswertung} entnommen werden sollen.

\subsection{Röntgenfluoreszenzanalyse}

\subsubsection{bereitgestellte Metalle}

\subsubsection{unbekannte Probe}

\subsection{Compton-Effekt}

\subsubsection{Energiekalibrierung des Detektors}

\subsubsection{Aufnahme der Spektren bei verschiedenen Winkeln}

\newpage
\printbibliography
\listoffigures
\listoftables
\end{document}
